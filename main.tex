\documentclass{kththesis}
\usepackage[utf8]{inputenc}
%\usepackage[pdftex]{graphicx}
\usepackage{amsmath}
\usepackage{amsfonts}
\usepackage{amssymb}
\usepackage[hyphens]{url}
\usepackage{hyperref}
\usepackage{siunitx}
\usepackage{pdflscape}
\usepackage{geometry}
\usepackage[toc]{glossaries} % https://www.overleaf.com/learn/latex/Glossaries
\usepackage{parskip} % empty line between paragraphs

% hyperref setup
\definecolor{darkgreen}{RGB}{0,180,0}
\hypersetup{
    colorlinks = true,
    linkbordercolor = {white},
    linkcolor = red,
    anchorcolor = black,
    citecolor = darkgreen,
    filecolor = cyan,
    menucolor = black,
    runcolor = cyan,
    urlcolor = magenta
}

% Bib setup
\usepackage{csquotes} % Recommended by biblatex
\usepackage[style=numeric,sorting=none,backend=biber]{biblatex}
\addbibresource{references.bib} % The file containing our references, in BibTeX format

\title{Experiments on automation of formal verification of devices at the binary level}
%\alttitle{}
\author{Thomas Lacroix}
\email{thomas.lacroix@insa-lyon.fr}
\supervisor{Mads Dam}
\examiner{TODO}
\hostcompany{Department of Theoretical Computer Science - KTH}
\programme{Master in Computer Science}
\school{Computer Science department - INSA Lyon}
\date{\today}

% Uncomment the next line to include cover generated at https://intra.kth.se/kth-cover?l=en
% \kthcover{kth-cover.pdf}

%%%%%%%%%%%%%%%%%%%%%%%%%%%%%%%%%%%%%%%%%%%%%%%%%%%%%%%%%%%%%%%%%%%%%%%%%%%%%%%%%%%%%%%%
%%%%%%%%%%%%%%%%%%%%%%%%%%%%%%%%%%%%%%%%%%%%%%%%%%%%%%%%%%%%%%%%%%%%%%%%%%%%%%%%%%%%%%%%
%% Glossary

\makeglossaries

%\gls{latex} or \Gls{latex}, \Glspl{formula}
%\acrlong{gcd} \acrshort{gcd} \acrfull{gcd}

\newglossaryentry{FreeRTOS} {
    name={FreeRTOS},
    description={FreeRTOS is a free and open-source market leading real time operating system for microcontrollers and small microprocessors. Website: \url{https://freertos.org/}}
}
\newglossaryentry{iot} {
    name={IoT},
    description={Internet Of Things. This is the idea of connecting every devices---such that temperature sensors, lamps, video cameras or wearable devices---to the Internet}
}
\newacronym{prosper}{PROSPER}{Provably Secure Execution Platforms for Embedded Systems}
\newglossaryentry{transpilation} {
    name={transpilation},
    description={Process carried-out by a \gls{transpiler}}
}
\newglossaryentry{transpiler} {
    name={transpiler},
    description={A transpiler, or source-to-source compiler, is a compiler that takes source code written in a programming language A, and producing an equivalent source code in a programming language B. A and B can be different languages or the same. They differ from traditional compilers in that compilers generally produce machine code}
}
\newacronym{holba}{HolBA}{HOL4 Binary Analysis Platform}
\newacronym{bap}{CMU BAP}{Carnegie Mellon University Binary Analysis Platform}
\newglossaryentry{BIR} {
    name={BIR},
    description={\acrshort{holba}'s machine-independent binary intermediate language}
}
\newacronym{isa}{ISA}{Instruction Set Architecture}

%% End of Glossary
%%%%%%%%%%%%%%%%%%%%%%%%%%%%%%%%%%%%%%%%%%%%%%%%%%%%%%%%%%%%%%%%%%%%%%%%%%%%%%%%%%%%%%%%
%%%%%%%%%%%%%%%%%%%%%%%%%%%%%%%%%%%%%%%%%%%%%%%%%%%%%%%%%%%%%%%%%%%%%%%%%%%%%%%%%%%%%%%%

\begin{document}

\frontmatter % titlepage, abstracts and TOC

\titlepage

\begin{abstract}
  With the advent of virtualization, more and more work is put into the verification of hypervisors. Being low level softwares, such verification should preferably be performed at binary level. Binary analysis platforms are being developed to help perform these proofs, but a lot of the work has to be carried out manually.
  
  In this thesis, we focus on the formal verification of a Network Interface Controller (NIC), more specifically we look at how to automate and reduce the boilerplate work from an existing proof. We base our work on the HolBA platform, its hardware-independent intermediate representation language BIR and supporting tools, and we experiment on how to perform this proof by leveraging existing tools.
  
  We first replaced the existing NIC model written in HOL4 to an equivalent one written using BIR, enabling the use of HolBA tools. Secondly, we developed some visualization tools to help navigate and gain some insight in the existing proof and its structure. Thirdly, we experimented with the use of Hoare triples in conjunction with an SMT solver to perform contract verification. Finally, we proved a simple contract written in terms of the formal NIC model on the BIR implementation of this model, unlocking the way of performing more complex proofs using the HolBA platform.
  
  \textit{Keywords}: binary analysis, formal verification, proof producing analysis, theorem proving
\end{abstract}

%\begin{otherlanguage}{swedish}
%  \begin{abstract}
%  \end{abstract}
%\end{otherlanguage}
\begin{otherlanguage}{french}
  \begin{abstract}
    Avec la démocratisation de la virtualisation, de plus en plus d'efforts sont consacrés à la vérification des hyperviseurs. S'agissant de logiciels de bas niveau, une telle vérification devrait de préférence être effectuée au niveau binaire. Des plates-formes d'analyse binaire sont en cours de développement pour aider à réaliser ces preuves, mais une grande partie du travail doit encore être effectuée manuellement.
      
    Dans cette thèse, nous nous concentrons sur la vérification formelle d'un Contrôleur d'Interface Réseau (NIC), plus spécifiquement sur la manière d'automatiser et de réduire le travial d'une preuve existante. Nous nous basons sur la plate-forme HolBA, son langage de représentation intermédiaire indépendant du matériel, BIR et ses outils de support, et nous nous intéressons à la manière de réaliser cette preuve en utilisant des outils existants.
    
    Nous avons d'abord remplacé le modèle NIC existant écrit en HOL4 par un modèle équivalent écrit en BIR, permettant ainsi l'utilisation des outils de HolBA. Deuxièmement, nous avons développé des outils de visualisation pour vous aider à naviguer et à mieux comprendre la preuve existante et sa structure. Troisièmement, nous avons expérimenté l'utilisation des triplets de Hoare en conjonction avec un solveur SMT pour effectuer une vérification par contrat. Enfin, nous avons prouvé un contrat simple écrit en termes du modèle formel du NIC sur l'implémentation de ce modèle en BIR, ouvrant la voie à la réalisation de preuves plus complexes avec la plate-forme HolBA.
    
    \textit{Mot-clés} : binary analysis, formal verification, proof producing analysis, theorem proving
  \end{abstract}
\end{otherlanguage}

\tableofcontents

\printglossaries

\mainmatter % Thesis content

\chapter{Introduction}
\textit{This chapter serves as an introduction to the degree project and presents the background of the work along with this thesis objective. Delimitations to the project and the choice of methodology are also discussed.}

TODOs:
\begin{itemize}
  \item Where to mention
  \begin{itemize}
  \item Git workflow for a team
  \item LogLib (and tracing in general)
  \item CI to track regressions + static analysis
  \end{itemize}
\end{itemize}

\section{Background}

\section{Intended readers}

\section{Thesis objective}

\section{Delimitations}

\section{Choice of methodology}

\chapter{Definitions and relevant theory}
\textit{This chapter intends to lay the concepts and theory that are essential to the reader in order to understand the problem that this degree project aims to explore.  This includes an overview of virtualization and hypervisors, a presentation of Interactive Theorem Proving and formal proofs, relevant theory about Hoare Triple and Weakest Precondition analysis, and an introduction of the \gls{holba} framework.}


\chapter{NIC model}
% * Présentation du modèle HOL4 du NIC (+ DepGraph)
% * Expérimentation avec Alice et Bob (+ CFG (+ interface))
% * Modèle BSL (-> prez de BSL)
%
% * Liberté de la méthode et question sur comment traduire le modèle
%
% * Impossibilité de tout représenter avec BIR (pourquoi?)
%   -> identification (> DepGraph) et fragmentation
%   -> transition vers la preuve


\chapter{Proving properties}

% * Rappels sur le WP existant dans HolBA
% * Explication de l'utilisation de SMT solver
%   - Ajout des arrays dans l'export existant
%   - Experiments with SMT (before and after array): without and with BIR
%   --- With BIR needed BIR EXP -> Words (why?)
%   ----- Non proof-producing + explain why (mention that PP has been done in the proof)
% * Non proof-producing WP lib
%   + easy
%   - proof-producing would have been feasible
%     -> but would have took more time, that's not the aim of the thesis
% * BIR PP because huge WP BIR expressions
% * WP tests
%   - simple
%   - Gauss analysis (how to handle loop, mult problem with SMT)
%   - memories
%   - cjmp bug?
%   - explain the simp (ie. the subst theory) + the exponential growth issue
% * Using the WP tool on simple properties on the BIR NIC model
%   - Remind about the impossibility to express certain properties using contracts
% * Explain the PP proof
%   - Whiteboard explanation
%   - Pen and paper proof
%   - About implementation
%   --- About PP BIR to words

\section{Overview of HolBA's WP tool}

\section{Using SMT solvers to prove contracts}

\section{Implementation of a non proof-producing WP library}
\subsection{Pretty-printing to visualize huge BIR expressions}
\subsection{Testing the library}

\section{Simple automatized proofs on the NIC model}

\section{Trustful analysis on the NIC model}


\chapter{Conclusions}

\section{Results}

\section{Discussion}

\section{Future work}

\printbibliography[heading=bibintoc]

\appendix

\chapter{Something Extra}

\tailmatter % back cover page

\end{document}